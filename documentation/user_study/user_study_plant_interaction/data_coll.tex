\section{Data collection}

To capture the participant's interactions with the plants, a collaborative approach was adopted, involving two researchers to provide dual perspectives.
Throughout the exploration phase, both researchers took notes, documenting the diverse ways in which participants engaged with the three distinct plants.
The researchers explicitly specified the plant involved in the interaction in order to extract special features related to a specific plant.

The written notes retrieved descriptions of participants' actions, movements and interactions.
The dual-observer strategy tends to reduce the potential biased.

At the beginning of the experiment, the \textit{Dypsis lutescens} was on the floor, the \textit{Dracaena} was on a chair and the \textit{Pachira Glabra} was on a table.
At the middle of the experiment, we switched the \textit{Dypsis lutescens} and the \textit{Pachira Glabra} to see if the participants would interact differently with the plants.
The set-up of the experiment is shown in Figure \ref{fig:setup_user_study}. 