\section{State of the art}

In the HCI world, the question of the human-plant interaction has already been asked. 
Seow and al. developed a framework that augment a \textit{Mimosa Pudica} \cite{seowPudicaFrameworkDesigning2022}. They designed and conducted a user study to test their framework
Their user study is following a script of question that are based on the experience with the already built framework.

A more technical approach is explained by Poupyrev and al. with the \textit{Botanicus Interacticus} framework \cite{poupyrev2012botanicus}.
The report does not present a user study but only the prototype for an art exhibition \cite{kasik2012hands}.
The information on the way to interpret the plant interaction is vague and only talk about machine learning.
 

R.F.M. worked on the interaction grade schoolers can develop with their plants \cite{ledo2019music}. 
The study tries to capture the human-plant link through sensors and also supervise the growing of the plants.
Contrary to Seow and al. \cite{seowPudicaFrameworkDesigning2022}, he is only using non-intrusive ways to capture the interaction.

The human-plant interaction and collaboration is studied in different field of research. 
For example in the psychologic domain, Elings studied the benefits of horticultural therapy which includes the interaction between patient and pants during therapy \cite{elings2006people}.

This study will be conducted using the human centered paradigm in the human-plant interface.
The reflection about the position of the plant or the human in the human-plant interaction is also a subject to be discuss.
Indeed, Loh and al. studied and criticize the human centered HCI that is reducing the plant to an actuator \cite{lohMorethanhumanTurnHumanplant2024}.